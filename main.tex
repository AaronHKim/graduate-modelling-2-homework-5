\documentclass[12pt,letterpaper]{article}
\usepackage{preamble}
\usepackage{siunitx}
\usepackage{adjustbox}
\usepackage{enumitem}
%%%%%%%%%%%%%%%%%%%%%%%%%%%%%%%%%%%%%%%%%%
%%%% Edit These for yourself
%%%%%%%%%%%%%%%%%%%%%%%%%%%%%%%%%%%%%%%%%%

\newcommand{\mathsym}[1]{{}}
\newcommand{\unicode}[1]{{}}

\newcommand\course{Graduate Modelling 2 MAT 5460}
\newcommand\hwnumber{4}
\newcommand\userID{Aaron Kim}
\def\changemargin#1#2{\list{}{\rightmargin#2\leftmargin#1}\item[]}
\let\endchangemargin=\endlist 

\theoremstyle{definition}
\newtheorem*{thm}{Theorem}

\renewcommand\thesection{\arabic{section}}
\renewcommand\thesubsection{\thesection.\arabic{subsection}}



\begin{document}
\section*{Problem 26}

Show that the Casoratian $W(n)$ given by:
\begin{equation}
    W(n) = \left|\begin{matrix}
        x_1(n) & x_2(n) &  \cdots & x_k(n) \\
        x_1(n+1) & x_2(n+1) & \cdots & x_k(n+1)\\
        \vdots & & & \vdots\\
        x_1(n+k-1) & x_2 (n+k-1) & \cdots & x_k(n+k-1)
    \end{matrix}\right|
\end{equation}
may be given by the formula:
\begin{equation*}
    W(n) = \left|\begin{matrix}
        x_1(n) & x_2(n) & \cdots & x_k(n)\\
        \Delta x_1(n) & \Delta x_2(n) & \cdots & \Delta x_k(n)\\
        \vdots & & & \vdots \\
        \Delta^{k-1}x_1(n) & \Delta^{k-1} x_2 (n) & \cdots & \Delta^{k-1} x_k(n)
    \end{matrix}\right|
\end{equation*}



\textbf{Attempt:}
\begin{changemargin}{.5cm}{0cm}

We know by definition that,  $\Delta x_i(n)=x_i(n+1)- x_i(n)$ , and that $\Delta x_i(n+k-1)=x_i(n+k)-x_i(n+k-1)$

\end{changemargin}

\newpage

\section*{Problem 27}

Consider the second order difference equation 
\begin{equation*}
    u(n+2) - \dfrac{n+3}{n+2}u(n+1) + \dfrac{2}{n+2}u(n) = 0
\end{equation*}

\begin{enumerate}[label = (\alph*)]
    \item show that $\displaystyle u_1(n)= \dfrac{2^n}{n!}$ is a solution of the equation.
    \item Use the above formula to find another solution $u_2(n)$ of the equation.
\end{enumerate}

\textbf{Attempt:}
\begin{changemargin}{.5cm}{0cm}

We know that $x^{(k)}$ is defined as follows:
\begin{equation*}
    x^{(k)}=x(x-1)(x-2)\cdots(x-k+1), \; k\in\mathbb{Z}^+
\end{equation*}
\begin{equation*}
    (x+1)^{(k)}=(x+1)(x)(x-1)\cdots(x-k+2), \; k\in\mathbb{Z}^+
\end{equation*}
This can be similarly defined as:
\begin{equation}\label{eq:deltafactorial}
    x^{(k)}=\dfrac{x!}{(x-k)!}=x(x-1)(x-2)\cdots(x-k+1) 
\end{equation}
and as:
\begin{equation}\label{eq:deltaprod}
    x^{(k)}=\prod_{j=1}^{k}x-j+1 = x(x-1)(x-2)\cdots(x-k+1) 
\end{equation}
\begin{equation*}
    (x+1)^{(k)}=\prod_{j=1}^{k-1}x-j+1 = (x+1)(x)(x-2)\cdots(x-k+2)
\end{equation*}
and from the lecture notes that:
\begin{align*}
    \Delta x^{(k)} &= (x+1)^{(k)}-x^{(k)}\\
                    &=(x+1)(x)(x-1)\cdots(x-k+2)-x(x-1)\cdots(x-k+2)(x-k+1)\\
                    &=(x)(x-1)\cdots(x-k+2)[x+1-(x-k+1)]\\
                    &=x(x-1)\cdots(x-k+2)k\\
                    &=kx^{(k-1)}
\end{align*}

\begin{enumerate}[label=(\roman*)]
    \item If we express the above as $\Delta^1 x^{(k)}$, then we can do the following:
    \begin{align*}
        \Delta\left[\Delta x^{(k)}\right] &= \Delta \left[ kx^{(k-1)} \right]\\
        &= k \Delta x^{(k-1)}\\
        &= k \left[(x+1)^{(k-1)} - x^{(k-1)}\right]\\
        &= k \Big[ (x+1)(x)(x-1)\cdots(x-(k-1)+2)\\
        &\quad- x(x-1)\cdots(x-(k-1)+2)(x-(k-1)+1)  \Big]\\
        &= k\Big[ (x)(x-1)\cdots(x-(k-1)+2)\Big] \Big( (x+1) -(x-(k-1)+1) \Big)\\
    \end{align*}
    The term on the left in square brackets is identically $x^{(k+2)}$ since:
    \begin{equation*}
        x^{(k-2)} = x(x-1)\cdots (x-(k-2)+1) = x(x-1)\cdots (x-k+3)
    \end{equation*}
    and the term on the right simplifies to:
    \begin{equation*}
        \Big( (x+1) -(x-(k-1)+1) \Big)= k-1
    \end{equation*}
    giving us:
    \begin{equation*}
        \Delta^2x^{(k)} = k(k-1)x^{(k-2)}
    \end{equation*}
    Now let's assume that this pattern holds up to some $(n-1)$, then:
    \begin{equation*}
        \Delta^{n-1} x^{(k)} = k(k-1)\cdots((k-(n-1)-1))x^{(k-(n-1)}
    \end{equation*}
    then:
    \begin{align*}
        \Delta\left[\Delta^{n-1}x^{(k)}\right] &= \Delta\left[k(k-1)\cdots(k-(n-1)-1)x^{(k-(n-1))}\right]\\
        &= \Delta\left[k(k-1)\cdots(k-n)x^{(k-n+1)}\right]\\
    \end{align*}
    We can express the quantity on the left inside the brackets as:
    \begin{equation}\label{eq:factorial}
        k(k-1)\cdots(k-n)= \dfrac{k!}{(k-n-1)!}
    \end{equation}
    and since it is a scalar quantity that does not depend on $x$ and is thus not effected by the $\Delta$ operator, we can move it outside. (since the $\Delta$ operator is linear)
    \begin{align*}
        \Delta\left[\Delta^{n-1}x^{(k)}\right] &= \dfrac{k!}{(k-n-1)!}\Delta x^{(k-n+1)}\\
        &= \dfrac{k!}{(k-n-1)!}\left[ (x+1)^{(k-n+1)}-x^{(k-n+1)}\right]
    \end{align*}
    We know that the structure of the operator as before in $\displaystyle \Delta^2 x^{(k)}$, the bracketed part will have the form from (\ref{eq:deltaprod}), where $(x+1)^{(k)}$ shifts both indicies down by one:
    \begin{equation*}
        \prod_{j=0}^{k-n} x-j+1 - \prod_{j=1}^{k-n+1} x-j+1
    \end{equation*}
    which simplifies to:
    \begin{equation*}
        \left[\prod_{j=1}^{k-n}x-j+1\right] [(x+1)-(x-(k-n+1)+1) ] = \left[\prod_{j=1}^{k-n}x-j+1\right] (k-n+1)
    \end{equation*}
    going back to what we had, we can substitute into the bracketed part: 
    \begin{align*}
        \Delta\left[\Delta^{n-1}x^{(k)}\right] &= \dfrac{k!}{(k-n-1)!}\left[ (x+1)^{(k-n+1)}-x^{(k-n+1)}\right]\\
        &= \dfrac{k!}{(k-n-1)!}\left[\prod_{j=1}^{k-n}x-j+1\right] (k-n+1)
    \end{align*}
    Simplifying further using (\ref{eq:deltaprod}) again we then get:
    \begin{equation}\label{eq:general}
        \Delta^nx^{(k)}=\dfrac{k!}{(k-n-1)!}x^{(k-n)} (k-n+1)
    \end{equation}
    \begin{equation*}
        \Delta^nx^{(k)}=k(k-1)\cdots(k-n)(k-n+1)x^{(k-n)} 
    \end{equation*}
    \item Using (\ref{eq:general}), if $n=k$:
    \begin{align*}
        \Delta^kx^{(k)}&=\dfrac{k!}{(k-k-1)!}x^{(k-k)} (k-k+1)\\
    \end{align*}
    and using (\ref{eq:factorial})
    \begin{align*}
        \Delta^kx^{(k)}&=\dfrac{k!}{(-1)!}\dfrac{x!}{(x-(k-k))!} (0+1)\\
        &= \dfrac{k!}{(-1)!} (1)(1)\\
        &=\dfrac{k!}{(-1)!}
    \end{align*}
        
    
    The term in brackets is zero, since the indicies are backwards. For the term on the left, we know that negative interger factorials have the form:
    \begin{equation*}
        n! = \dfrac{(-1)^{-n-1}}{(-n-1)!}
    \end{equation*}
    and that $-1! = 1$, so this gives us:
    \begin{equation*}
        \Delta x^{(k)} = k!
    \end{equation*}
\end{enumerate}

\end{changemargin}
\newpage


\section*{Problem 28}

Write the general solution of the difference equation:
\begin{equation*}
    (E-3)^2(E^2+4)x(n)=0
\end{equation*}



\textbf{Attempt:}

    \begin{changemargin}{.5cm}{0cm}
        since $\Delta = E-I$,
        \begin{align*}
            \Delta\left[\frac{x(n)}{y(n)}\right] &= (E-I)\left[\frac{x(n)}{y(n)}\right]\\
            &= E\left[\frac{x(n)}{y(n)}\right] -I\left[\frac{x(n)}{y(n)}\right] \\
            &=\left[\frac{x(n+1)}{y(n+1)}\right] -\left[\frac{x(n)}{y(n)}\right] \\
            &= \left[\frac{x(n+1)y(n)-x(n)y(n+1)}{y(n)y(n+1)}\right] \\
            &=\left[\frac{x(n+1)y(n)+[y(n)x(n)-y(n)x(n)]-x(n)y(n+1)}{y(n)y(n+1)}\right] \\
            &=\left[\frac{x(n+1)y(n)-y(n)x(n)-x(n)y(n+1)+y(n)x(n)]}{y(n)y(n+1)}\right] \\
            &=\left[\frac{y(n)[x(n+1)-x(n)]-x(n)[y(n+1)-y(n)]}{y(n)y(n+1)}\right] \\
            &=\left[\frac{y(n)\Delta x(n)-x(n)\Delta y(n)}{y(n)Ey(n)}\right]
        \end{align*}

    \end{changemargin}





\newpage

%

\section*{Problem 29}

Find a homogenous difference equation whose solution is:
\begin{equation*}
    y(n) = 2^{n-1}-5^{n+1}
\end{equation*}


\textbf{Attempt:}

\begin{changemargin}{.5cm}{0cm}
    \begin{enumerate}
        \item    
   if $X_1(n)$ and $X_2(n)$ are solutions to \ref{eq:p24}, then:
   \begin{equation*}
    X_1(k+n) + p_1(n) X_1(n+k-1)+\dots +p_k(n)X_1(n)=0
   \end{equation*}
   \begin{equation*}
    X_2(k+n) + p_1(n) X_2(n+k-1)+\dots +p_k(n)X_2(n)=0
   \end{equation*}
   So if we plug in $x(n) = X_1(n) + X_2(n)$ for (\ref{eq:p24})
   \begin{align*}
    x(n)=&X_1(k+n)+X_2(k+n) + p_1(n) X_1(n+k-1)+p_1(n)X_2(n+k-1)\\
    &\quad+\dots +p_k(n)X_1(n)+p_k(n)X_2(n)\\
    &=X_1(k+n) + p_1(n) X_1(n+k-1)+\dots +p_k(n)X_1(n) \\
    &\quad+X_2(k+n) + p_1(n) X_2(n+k-1)+\dots +p_k(n)X_2(n)\\
    &=0 +0
   \end{align*}
    But this is just equal to the two given solutions meaning the sum of each (when we split them) is zero, so this is also a solution.
    \item  Doing something similar:
    \begin{align*}
        \tilde{x}(n) &= aX_1(k+n) + p_1(n) aX_1(n+k-1)+\dots +p_k(n)aX_1(n)\\
        &=a(X_1(k+n) + p_1(n) X_1(n+k-1)+\dots +p_k(n)X_1(n))\\
        &=a(0)\\
        &=0
    \end{align*}
    So this is also a solution.
\end{enumerate}
\end{changemargin}

\newpage

\section*{Problem 30}

Find a partic


    
  \textbf{Attempt:}
    \begin{changemargin}{.5cm}{0cm}
    
    \begin{enumerate}[label=(\alph*)]
    \item let $n=1$
    \begin{equation*}
        \begin{vmatrix}
            1 & 1 & 3\\
            -2 & 2 & 3\\
            4 & 4 & 3
        \end{vmatrix}=
        1\begin{vmatrix}
            2 & 3\\
            4 & 3
        \end{vmatrix}
        -(-2)\begin{vmatrix}
            1 & 3\\
            4 & 3
        \end{vmatrix}
        +4\begin{vmatrix}
            1 & 3\\
            2 & 3
        \end{vmatrix}
    \end{equation*}
    \begin{equation*}
    = 6-12+ 6-24 + 12-24 = -12-6-18 = -36
   \end{equation*}
   since this $\neq 0$ these are linearly independent
    \item 
    \begin{equation*}
        \begin{vmatrix}
            0 & 1 & 3\\
            0 & 2 & 3\\
            0 & 4 & 3
        \end{vmatrix}=
        0\begin{vmatrix}
            2 & 3\\
            4 & 3
        \end{vmatrix}
        -0\begin{vmatrix}
            1 & 3\\
            4 & 3
        \end{vmatrix}
        +0\begin{vmatrix}
            1 & 3\\
            2 & 3
        \end{vmatrix} = 0
    \end{equation*}
    So this is linearly dependent.
    \item 
    \begin{equation*}
        \begin{vmatrix}
            1 &  3 & 4\\
            2 &  3 & 8\\
            4 &  3 & 16
        \end{vmatrix}=
        1\begin{vmatrix}
            3 & 8\\
            3 & 16
        \end{vmatrix}
        -2\begin{vmatrix}
            3 & 4\\
            3 & 16
        \end{vmatrix}
        +4\begin{vmatrix}
            3 & 4\\
            3 & 8
        \end{vmatrix}
    \end{equation*}
    \begin{equation*}
    = 48-24+ -96+24 + 96-48 = 0 
   \end{equation*}
   So this is linearly dependent 
\end{enumerate}


    \end{changemargin}


\end{document}